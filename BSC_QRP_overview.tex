\documentclass[11pt]{article}
\usepackage[margin=1in]{geometry}
\usepackage{amsmath,amssymb,amsfonts}
\usepackage{fancyhdr}

\pagestyle{fancy}
\fancyhead{}
\fancyfoot{}
\fancyhead[L]{\slshape \MakeUppercase{Quantitative Research Projects}}
\fancyhead[R]{\slshape {BETA SIGMA CLUB}}
\fancyfoot[C]{\thepage}


\begin{document}

\begin{titlepage}
\begin{center}
\vspace*{1cm}
\vfill
\line(1,0){400}\\[1mm]
\huge{\textbf{Quantitative Research Projects}}\\[3mm]
\Large{\textbf{By BETA SIGMA CLUB}}\\[1mm]
\line(1,0){400}\\[3mm]
\vfill
\today

\end{center}
\end{titlepage}

\section{Introduction to Quantitative Research Projects}
The main goal of the Quantitative Research Projects offered by the BETA SIGMA CLUB is giving its' members the opportunity to apply their theoretical knowledge from Physics, Mathematics and other STEM-fields to quantitative financial (research) problems. We hope that this will inspire and enable them to pursue a career in quantitative finance. Next to that, the participants will also develop research and writing skills, learn to apply the scientific method and receive feedback from professionals and academics on the validity and accuracy of their work. 


\section{Practical Arrangements}
BETA SIGMA CLUB members are invited to send over their preferred project topics (up to 3 topics) to \textit{jos@betasigmaclub.com} by the first of November (1/11/2022). Depending on their choice a date for a feedback moment in the second semester will be communicated. By then a quantitative professional or academic will have reviewed their paper and have prepared some insightful comments. 

Groups should ideally have between 2 and 4 members, however working individually is also encouraged. If you currently don't have or know any group members, we will communicate available team mates after receiving your list of preferred topics. Please also mention the degree you are currently enrolled in, this will enable us to form interdisciplinary groups. This reflects the way quantitative research teams are actually formed in a professional context, where quants with different backgrounds team-up together to solve complex financial problems. 

Each team will be invited to a GitHub workspace as well. So be sure to mention your GitHub username or the e-mail you will use to make a GitHub account. For each project the following things will be provided in these GitHub workspaces: some raw data usefull to the research, a LaTeX template for the paper and some relevant literature.

\section{Project Proposals}

\subsection{The relationship between Entropy and the VIX and S\&P500 indices}

\subsection{Geometric Brownian Motion (GBM) applied in Physics and Stock Price Predictions}

\subsection{Numerical Approximations of the Feynman-Kac formula}

\subsection{Applying Decisions Trees to the Short-Rate model}

\subsection{Fitting Multivariate Normal and Multivariate t-distributions to returns of a portfolio of assets}

\subsection{Copulas and their applications}

\subsection{Stochastic Differential Equations in Physics and Finance}

\subsection{$\Huge{\sigma}$-Algebra in Mathematics and Finance}

\subsection{Time Series and Stochastic Processes}

\end{document} 